\chapter*{Homework 2}
\section*{Problem 1}
Suppose that the cumulative distribution function (CDF) of a random variable $F: \mathbb{R} \rightarrow \mathbb{R}$ is strictly increasing and continuous. Let $U$ be a random variable with the uniform distribution on $(0,1)$ and define
$$
X:=F^{-1}(U) .
$$
Show that $X$ has CDF equal to $F$.
This exercise shows us how to construct a random variable with given distribution, assuming that we have a uniform random variable.
\begin{proof}
Since $F$ is strictly increasing and continuous, it has an inverse function $F^{-1}$ on its range, and $F^{-1}$ is also strictly increasing.

For any $x \in \mathbb{R}$, we have
\[
\{X \le x\}=\{F^{-1}(U)\le x\}.
\]
Since $F^{-1}$ is strictly increasing, the above is equivalent to
\[
\{U \le F(x)\}.
\]
Therefore
\[
\mathbb{P}(X \le x)=\mathbb{P}(U \le F(x)).
\]
Since $U \sim \mathrm{Unif}(0,1)$ and for a CDF we have $F(x)\in[0,1]$, we get
\[
\mathbb{P}(U \le F(x)) = F(x).
\]
Thus for all $x$, the CDF of $X$ satisfies $\mathbb{P}(X \le x)=F(x)$, i.e., the CDF of $X$ equals $F$.
\end{proof}


\section*{Problem 2}
A gas station fills its tank completely once a week. 
Let the weekly sales volume (in thousands of liters) be a random variable with density
$$
f(x)= \begin{cases}a(1-x)^4, & x \in(0,1), \\ 0, & \text { otherwise } .\end{cases}
$$
Find the constant $a$. What should be the tank capacity 
so that the probability of running out of fuel during a given week is $1 / 100$ ?
\begin{solution}
(1) Since the density integrates to 1,
\[
1=\int_{-\infty}^{\infty} f(x)\,dx=\int_0^1 a(1-x)^4\,dx
= a \int_0^1 (1-x)^4\,dx.
\]
Let $u=1-x$, then $\int_0^1 (1-x)^4 dx=\int_0^1 u^4 du=\frac{1}{5}$, so $a\cdot \frac{1}{5}=1$, which gives
\[
a=5.
\]

(2) Let the tank capacity be $c$ (in thousands of liters). Running out of fuel in a week occurs when sales exceed $c$, i.e., the event $\{X>c\}$. We need
\[
\mathbb{P}(X>c)=\frac{1}{100}.
\]
Since $a=5$,
\[
\mathbb{P}(X>c)=\int_c^1 5(1-x)^4\,dx.
\]
Again let $u=1-x$, then when $x=c$ we have $u=1-c$, when $x=1$ we have $u=0$, so
\[
\int_c^1 5(1-x)^4 dx = 5\int_{1-c}^{0} u^4(-du)=5\int_0^{1-c} u^4 du
=5\cdot \frac{(1-c)^5}{5}=(1-c)^5.
\]
Therefore
\[
(1-c)^5=\frac{1}{100} \quad \Longrightarrow \quad 1-c=\left(\frac{1}{100}\right)^{1/5}=100^{-1/5}=10^{-2/5},
\]
thus
\[
c=1-10^{-2/5}.
\]
So the tank capacity should be $\bigl(1-10^{-2/5}\bigr)$ thousand liters, i.e., $1000\bigl(1-10^{-2/5}\bigr)$ liters.
\end{solution}





\section*{Problem 3}
Let the random variable $X$ have density
$$
f_X(x)= \begin{cases}\frac{1}{2 x^2}, & |x| \geq 1, \\ 0, & |x|<1 .\end{cases}
$$
Find the probability density function of $Y:=X^2$ and 
compute the probability $\mathbb{P}(2 Y+3 \leq 10)$.
\begin{solution}
Since $Y=X^2$, we have $Y\ge 1$. For $y\ge 1$, the transformation $y=x^2$ has two preimages $x=\sqrt{y}$ and $x=-\sqrt{y}$, and
\[
\left|\frac{dx}{dy}\right|=\frac{1}{2\sqrt{y}}.
\]
Therefore
\[
f_Y(y)= f_X(\sqrt{y})\frac{1}{2\sqrt{y}} + f_X(-\sqrt{y})\frac{1}{2\sqrt{y}}, \quad y\ge 1.
\]
Since $f_X(\pm \sqrt{y})=\frac{1}{2(\sqrt{y})^2}=\frac{1}{2y}$, we get
\[
f_Y(y)=\frac{1}{2y}\cdot\frac{1}{2\sqrt{y}}+\frac{1}{2y}\cdot\frac{1}{2\sqrt{y}}
= \frac{1}{2y^{3/2}}, \quad y\ge 1,
\]
and $f_Y(y)=0$ when $y<1$. That is,
\[
f_Y(y)=
\begin{cases}
\dfrac{1}{2y^{3/2}}, & y\ge 1,\\[6pt]
0, & y<1.
\end{cases}
\]

Next,
\[
\mathbb{P}(2Y+3\le 10)=\mathbb{P}\!\left(Y\le \frac{7}{2}\right)
= \int_{1}^{7/2} \frac{1}{2y^{3/2}}\,dy.
\]
Evaluating the integral:
\[
\int \frac{1}{2}y^{-3/2}dy=\frac{1}{2}\cdot\left(-2y^{-1/2}\right)=-y^{-1/2},
\]
therefore
\[
\mathbb{P}\!\left(Y\le \frac{7}{2}\right)
=\left[-y^{-1/2}\right]_{1}^{7/2}
=1-\frac{1}{\sqrt{7/2}}
=1-\sqrt{\frac{2}{7}}.
\]
\end{solution}






\section*{Problem 4}
Let the random variable $X$ have density $f$, 
which is symmetric about $\mu \in \mathbb{R}$, 
that is, $f(\mu+x)= f(\mu-x)$, for all $x \in \mathbb{R}$. 
Show that $\mathbb{P}(X \leq \mu)=\mathbb{P}(X \geq \mu)$. 
If in addition $\mathbb{E}|X|<\infty$, show that $\mathbb{E}(X)=\mu$.
 Can you use this observation if $X \sim N(0,1)$ ?
\begin{proof}
(1) Since $X$ has density $f$, we have $\mathbb{P}(X=\mu)=0$, and
\[
\mathbb{P}(X\le \mu)=\int_{-\infty}^{\mu} f(x)\,dx,\qquad
\mathbb{P}(X\ge \mu)=\int_{\mu}^{\infty} f(x)\,dx.
\]
For the first integral, substitute $x=\mu-t$, then $dx=-dt$. When $x\to -\infty$ we have $t\to \infty$, when $x=\mu$ we have $t=0$, so
\[
\int_{-\infty}^{\mu} f(x)\,dx
=\int_{\infty}^{0} f(\mu-t)(-dt)
=\int_{0}^{\infty} f(\mu-t)\,dt.
\]
By symmetry $f(\mu-t)=f(\mu+t)$, so
\[
\int_{0}^{\infty} f(\mu-t)\,dt=\int_{0}^{\infty} f(\mu+t)\,dt
=\int_{\mu}^{\infty} f(x)\,dx.
\]
Therefore $\mathbb{P}(X\le \mu)=\mathbb{P}(X\ge \mu)$.

(2) If $\mathbb{E}|X|<\infty$, then $\mathbb{E}|X-\mu|<\infty$, and
\[
\mathbb{E}(X-\mu)=\int_{-\infty}^{\infty} (x-\mu)f(x)\,dx.
\]
Let $t=x-\mu$, then
\[
\mathbb{E}(X-\mu)=\int_{-\infty}^{\infty} t\, f(\mu+t)\,dt.
\]
By symmetry $f(\mu+t)=f(\mu-t)$, the function $g(t):=t f(\mu+t)$ is odd, since
\[
g(-t)=(-t)f(\mu-t)=(-t)f(\mu+t)=-g(t).
\]
Under the condition that the expectation exists (absolutely integrable), the integral of an odd function over a symmetric interval is 0, so
\[
\mathbb{E}(X-\mu)=0 \quad \Longrightarrow \quad \mathbb{E}(X)=\mu.
\]

(3) If $X\sim N(0,1)$, its density is symmetric about $\mu=0$ and $\mathbb{E}|X|<\infty$, so we can apply the above result to get $\mathbb{E}(X)=0$.
\end{proof}





\section*{Problem 5}
An airline has observed that $5 \%$ of ticket holders do not show up for their flight. Today's flight has an airplane with 200 seats, and the airline has sold 203 tickets.
What is the probability that the airline will not be able to accommodate a ticketed passenger? Assume that, for each passenger $i$, 
the event $A_i$ that passenger $i$ shows up is independent of all others,
for $1 \leq i \leq 203$.
\begin{solution}
Let the random variable $S$ denote the number of passengers who actually show up among the 203 ticket holders. Then $S\sim \mathrm{Bin}(n=203,p=0.95)$.

The airline cannot accommodate ticketed passengers if and only if the number of passengers who show up exceeds the 200 seats, i.e., the event $\{S\ge 201\}$. Therefore the required probability is
\[
\mathbb{P}(S\ge 201)=\sum_{k=201}^{203} \binom{203}{k}(0.95)^k(0.05)^{203-k}.
\]
Numerically,
\[
\mathbb{P}(S\ge 201)\approx 0.002058,
\]
which is approximately $0.2058\%$.
\end{solution}


\section*{Problem 6}
Consider a sequence of tosses of a fair die. We continue tossing until both outcomes 3 and 4 have appeared at least once. For example, 
one possible sequence of results is
$$
5,1,1,4,6,5,4,2,6,3,
$$
and we then stop. Let $X$ be the number of tosses required (in this example, $X=10$ ).
What is the expected value of the random variable $X$ ?
\begin{solution}
We use states to characterize the appearance of $\{3,4\}$:

Let $E_0$ be the expected number of tosses needed from the initial state (neither 3 nor 4 has appeared) to the end.
Let $E_1$ be the expected number of tosses needed from the state where one of them has appeared (only 3 or only 4 has appeared) to the end.

First, find $E_1$. In state 1, each toss has probability $1/6$ of getting the missing number and thus ending, and probability $5/6$ of remaining in state 1. Therefore
\[
E_1 = 1 + \frac{5}{6}E_1 + \frac{1}{6}\cdot 0
\quad \Longrightarrow \quad
E_1 = 6.
\]

Next, find $E_0$. In state 0, each toss:
has probability $2/6$ of getting 3 or 4 and thus entering state 1,
has probability $4/6$ of getting other numbers and thus remaining in state 0. So
\[
E_0 = 1 + \frac{4}{6}E_0 + \frac{2}{6}E_1.
\]
Substituting $E_1=6$:
\[
E_0 = 1 + \frac{4}{6}E_0 + \frac{2}{6}\cdot 6
= 1 + \frac{4}{6}E_0 + 2.
\]
Rearranging gives
\[
\left(1-\frac{4}{6}\right)E_0 = 3
\quad \Longrightarrow \quad
\frac{2}{6}E_0 = 3
\quad \Longrightarrow \quad
E_0 = 9.
\]
Therefore $\mathbb{E}[X]=9$.
\end{solution}