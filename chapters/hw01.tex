\chapter*{Homework 1}



\section*{Problem 1}
Let $n \in \mathbb{N}$.
\begin{itemize}
    \item[(a)] Show that $2^n=\sum_{k=0}^n\binom{n}{k}$. Given that a set of $n$ elements has $2^n$ subsets, what is the combinatorial interpretation of this equality?
    \item[(b)] Show that $$\sum_{k \text { odd }, 0 \leq k \leq n}\binom{n}{k}=\sum_{k \text { even }, 0 \leq k \leq n}\binom{n}{k} $$
    \item[(c)] Show that $$\binom{2 n}{n}=\sum_{k=0}^n\binom{n}{k}^2 $$
\end{itemize}
Hint: You may use $\binom{n}{k}^2=\binom{n}{k}\binom{n}{n-k}$.
\begin{proof}
    \begin{itemize}
        \item[(a)] 
By the binomial theorem, we have:
\[
(1+1)^n = \sum_{k=0}^n \binom{n}{k} 1^k 1^{n-k} = \sum_{k=0}^n \binom{n}{k}
\]
The combinatorial interpretation of this equality is that:
Let $S_k$ be the collection of all subsets that have size $k$.
For collection $S_k$, its size is $\binom{n}{k}$ since it represents choosing $k$ elements from $n$ elements without regard to order.\\
Therefore the total number of subsets of $S$ is:
\[
| \mathcal{P}(S) |=\sum_{k=0}^n |S_k|  =\sum_{k=0}^n \binom{n}{k} =2^n
\]        
        \item[(b)] 
Using the binomial theorem, we have:
\[
0 = (1-1)^n = \sum_{k=0}^n \binom{n}{k} (-1)^k 1^{n-k} = \sum_{k=0}^n \binom{n}{k} (-1)^k
\]
Thus:
\begin{align}
    &\quad\quad\quad\quad \sum_{k=0}^n \binom{n}{k} (-1)^k = 0 \nonumber \\
    &\implies \sum_{0\leq k \leq n,\, k \text{ odd} } \binom{n}{k} (-1)
    + \sum_{0\leq k \leq n,\, k \text{ even} } \binom{n}{k} (1) = 0 \nonumber \\
    &\implies \sum_{0\leq k \leq n,\, k \text{ odd} } \binom{n}{k}
    = \sum_{0\leq k \leq n,\, k \text{ even} } \binom{n}{k}
\end{align}
        \item[(c)] 
We prove by combinatorial argument.\\
Let $S$ be a setwith $2n$ distinct elements. The number of ways to choose a subset $P$ containing $n$ elements is $\binom{2n}{n}$.\\
In another way: 
We can first arbitrarily divide the $2n$ distinct elements into two groups: group $A$ and group $B$, each containing $n$ elements:
\[
S = A \sqcup B
\]
And fix the two groups. \\
For any subset $P$ of the $2n$ elements with size $n$, some of them are from group $A$, and the rest of them are from group $B$. \\
Let $k$ be the number of elements of $P$ that are chosen from $A$, then the number of elements chosen from $B$ must be $n-k$.\\
Note the number of ways to choose $k$ elements from $A$ is $\binom{n}{k}$, and the number of ways to choose $n-k$ elements from $B$ is $\binom{n}{n-k}$. \\
Therefore, the total number of ways to get $P$ from $S=A \sqcup B$ with $k$ elements from $A$ is $\binom{n}{k}\binom{n}{n-k} = \binom{n}{k}^2$.\\
Thus, summing over all possible values of $k=0,1,\dots,n$, the number of ways to choose $n$ elements from $S$ i.e. the number of ways to get $P$ from $S$, is $$\sum_{k=0}^n \binom{n}{k}^2$$
Thus, we obtain $$\binom{2n}{n}=\sum_{k=0}^n \binom{n}{k}^2$$ as desired.
    \end{itemize}
\end{proof}


\section*{Problem 2}
We roll a fair die three times and record the outcomes $a, b, c \in\{1,2,3,4,5,6\}$. What is the probability that the equation $a x^2+b x+c=0$ does not have solutions in the real numbers?
\begin{solution}
    The equation $a x^2+b x+c=0$ does not have solutions in the real numbers iff the the discriminant is negative, i.e. $\Delta=b^2-4 a c<0$.\\
    Total possible equations is $6^3 = 216$. For each $b$, the total possible $(a,c)$ pairs are $36$. We can calculate the number of $(a,c)$ pairs that satisfy the condition case by case.
\begin{itemize}
\item For $b=1$: $4ac > 1$ holds for all $(a,c)$.
\item For $b=2$: $4ac > 4 \implies ac > 1$, which excludes only $(1,1)$.
\item For $b=3$: $4ac > 9 \implies ac \ge 3$ since they are integers, so excluding $(1,1),(1,2),(2,1)$ ($3$ cases).
\item For $b=4$: $4ac > 16 \implies ac \ge 5$, excluding: $(1,3),(1,4),(2,2),(3,1),(4,1)$ besides the previous case, thus $8$ cases excluded.
\item For $b=5$: $4ac > 25 \implies ac \ge 7$, excluding: $(1,5),(1,6),(2,3),(3,2),(5,1),(6,1)$ besides the previous case, thus $14$ cases excluded.
\item For $b=6$: $4ac > 36 \implies ac \ge 10$, excluding: $(2,4),(3,3),(4,2)$ besides the previous case, thus $17$ cases excluded.
\end{itemize}
Thus, the total number of triples for which the discriminant is not negative (exlcuded) is
\[
1+ 3 + 8 + 14 + 17 = 43
\]
Therefore, the desired probability is
\[
1-\mathbb{P}(\text{the equation has solutions in the real numbers}) = 1-\frac{43}{216} = \frac{173}{216}
\]
\end{solution}


\section*{Problem 3}
An ant starts at the origin $(0,0)$ on the integer lattice.
At each step it moves either one unit to the right or one unit upward, each with probability $\frac{1}{2}$. 
The ant continues moving until it reaches the point $(205,200)$.\\
What is the probability that the ant visits the point $(105,100)$ at some time during its journey?\\
Hint: Start by counting the number of paths from $(0,0)$ to $(205,200)$.
\begin{solution}
Any path from $(0,0)$ to $(205,200)$ must consist of $205$ steps to the right and $200$ steps upward, for a total of $405$ steps. 
So a path is uniquely determined by the choice of 205 steps to the right (which is equivalent to the choice of 200 steps upward).\\
Thus total number of paths from $(0,0)$ to $(205,200)$ is
\[
N = \binom{405}{205}
\]

A path passes through the point $(105,100)$ if and only if it first goes from $(0,0)$ to $(105,100)$ and then from $(105,100)$ to $(205,200)$.\\
Thus the number of such paths is the product of the number of paths from $(0,0)$ to $(105,100)$ and the number of paths from $(105,100)$ to $(205,200)$, by the fundamental counting principle.
For the same reason as deciding the number of total paths from $(0,0)$ to $(205,200)$, the number of paths from $(0,0)$ to $(105,100)$ is
\[
N_1 = \binom{205}{105}
\]
And similarly, the number of paths from $(105,100)$ to $(205,200)$ is
\[
N_2 = \binom{200}{100}
\]
Note that from a point to another point, all such paths are equally likely to be chosen. Therefore, the desired probability is
\[
\mathbb{P}(\text{path passes through $(105,100)$}) = \frac{\binom{205}{105}\binom{200}{100}}{\binom{405}{205}}
\]
\end{solution}

\section*{Problem 4}
From a lottery containing $n$ tickets numbered $1,2, \ldots, n$, a ticket is drawn, its number is recorded, and then it is returned to the lottery. This process is repeated $k \geq 3$ times. Find the probabilities of the following events:
\begin{itemize}
    \item[(a)] Ticket 1 is selected at least once.
    \item[(b)] Tickets 1, 2, and 3 are each selected at least once.
\end{itemize}
\begin{solution}
\begin{itemize}
    \item[(a)] 
 Let $E$ be the event that ticket $1$ is selected at least once.
 \begin{align*}
    \mathbb{P}(E) &= 1 - \mathbb{P}(\text{ticket $1$ is never selected in $k$ draws}) \\
    &= 1 - \left(\frac{n-1}{n}\right)^k \tag*{\text{(by independence of each draw)}}
 \end{align*}
    \item[(b)] 
Let $F$ be the event that tickets $1,2,3$ are each selected at least once. \\
For $i=1,2,3$, let
\[
A_i:=\{\text{ticket } i \text{ is never selected in the } k \text{ draws}\}
\]
Thus 
\[
P(F) = 1 - P(A_1 \cup A_2 \cup A_3)
\]
By the principle of inclusion-exclusion,
\[
P(A_1 \cup A_2 \cup A_3) = P(A_1) + P(A_2) + P(A_3) - P(A_1 \cap A_2) - P(A_1 \cap A_3) - P(A_2 \cap A_3) + P(A_1 \cap A_2 \cap A_3)
\]
Since similar to part (a), we have:\(\mathbb{P}(A_i)=\left(\frac{n-1}{n}\right)^k\), \(\mathbb{P}(A_i\cap A_j)=\left(\frac{n-2}{n}\right)^k\), \(\mathbb{P}(A_1\cap A_2\cap A_3)=\left(\frac{n-3}{n}\right)^k\), we then calculate:
\begin{align*}
\mathbb{P}(F) &= 1 - \binom{3}{1}\left(\frac{n-1}{n}\right)^k + \binom{3}{2}\left(\frac{n-2}{n}\right)^k - \binom{3}{3}\left(\frac{n-3}{n}\right)^k\\
&=1-3\left(\frac{n-1}{n}\right)^k+3\left(\frac{n-2}{n}\right)^k-\left(\frac{n-3}{n}\right)^k
\end{align*}
\end{itemize}
\end{solution}



\section*{Problem 5}
In a house, drawer $S_1$ contains 3 gold coins and 3 silver coins, 
while drawer $S_2$ contains 3 gold coins and 6 silver coins. 
A thief (in the dark) randomly opens one drawer and then randomly takes two coins from it.
\begin{itemize}
    \item[(a)] What is the probability that both coins are gold?
    \item[(b)] If it is discovered (upon his arrest) that he has stolen two gold coins, what is the probability that he opened drawer $S_1$ ?
\end{itemize}
\begin{solution}

The thief chooses a drawer uniformly at random, so for each pick, $\mathbb{P}(\text{$S_1$ is chosen}) = \mathbb{P}(\text{$S_2$ is chosen}) = \frac12$. 
Given a drawer, he draws two coins without replacement.
\begin{itemize}
\item[(a)] Using the law of total probability,
\begin{align*}
\mathbb{P}(\text{two gold}) &= \mathbb{P}(\text{two gold}\mid S_1 \text{ is chosen})\mathbb{P}(\text{drawer $S_1$}) + \mathbb{P}(\text{two gold}\mid S_2 \text{ is chosen})\mathbb{P}(\text{drawer $S_2$}) \\
&= \frac{1}{2} \cdot \frac{\binom{3}{2}}{\binom{6}{2}} + \frac{1}{2} \cdot \frac{\binom{3}{2}}{\binom{9}{2}} \\
&= \frac{1}{2} \left( \frac{3}{15} + \frac{3}{36} \right) \\
&= \frac{36 + 15}{360}\\
&= \frac{17}{120}
\end{align*}
\item[(b)] 
Let $G$ be the event that the thief stole two gold coins. By Bayes' rule,
\[
\mathbb{P}(S_1\mid G)=\frac{\mathbb{P}(G\mid S_1)\mathbb{P}(S_1)}{\mathbb{P}(G)}
\]
Since we have $\mathbb{P}(G\mid S_1)=\frac{\binom{3}{2}}{\binom{6}{2}}=\frac15$, $\mathbb{P}(S_1)=\frac12$, and $\mathbb{P}(G)=\frac{17}{120}$ from part (a), we get:
\[
\mathbb{P}(S_1\mid G)=\frac{\frac15\cdot \frac12}{\frac{17}{120}}
=\frac{12}{17}
\]
\end{itemize}
\end{solution}


\section*{Problem 6}
Let $A$ and $B$ be events of a probability space with $\mathbb{P}(A)>0$. Show that:
\begin{itemize}
    \item[(a)] $\mathbb{P}(A \cup B)>0$ and $\mathbb{P}(A \cap B \mid A \cup B) \leq \mathbb{P}(A \cap B \mid A)$.
    \item[(b)] $\mathbb{P}(B \mid B \cup A) \geq \mathbb{P}(B \mid A)$.
\end{itemize}
\begin{proof}

\begin{itemize}
\item[(a)] Since $A\subseteq A\cup B$, we have by monotonicity of probability measure:
\[
\mathbb{P}(A\cup B)\ge \mathbb{P}(A)>0
\]
Also, since $A\cap B\subseteq A$ and $\mathbb{P}(A\cup B)\ge \mathbb{P}(A)>0$, 
both conditional probabilities below are well-defined. 
Then
\begin{align*}
\mathbb{P}(A\cap B\mid A\cup B)
&=\frac{\mathbb{P}((A\cap B) \cap (A\cup B))}{\mathbb{P}(A\cup B)} \\
&= \frac{\mathbb{P}(A\cap B)}{\mathbb{P}(A\cup B)} \\
&\le \frac{\mathbb{P}(A\cap B)}{\mathbb{P}(A)}\\
&=\mathbb{P}(A\cap B\mid A)
\end{align*}
This finishes the proof.
\item[(b)] 
Let \(x:=\mathbb{P}(A\cap B)\), \(y:= \mathbb{P}(A\setminus B)\), \(z:= \mathbb{P}(B\setminus A)\).\\
so $x,y,z\ge 0$ by non-negativity of probability measure.\\
And since \[
A = (A\cap B) \sqcup (A\setminus B)
\]
Thus, we have: 
\[
\mathbb{P}(A)=\mathbb{P}(A\cap B)+\mathbb{P}(A\setminus B)=x+y
\]
By similar reason, we have:
\[
\mathbb{P}(A\cup B)=x+y+z,\quad \mathbb{P}(B)=x+z
\]
Thus we have:
\[
\mathbb{P}(B\mid A\cup B)= \frac{\mathbb{P}(B \cap (A\cup B))}{\mathbb{P}(A\cup B)} = \frac{\mathbb{P}(B)}{\mathbb{P}(A\cup B)}=\frac{x+z}{x+y+z}
\]
and 
\[
\mathbb{P}(B\mid A)=\frac{\mathbb{P}(A\cap B)}{\mathbb{P}(A)}=\frac{x}{x+y}
\]
Note the two probabilities are well-defined since $x+y= \mathbb{P}(A)>0$ (and so $x+y+z>0$).\\
Now it remains to show that:
\[
\frac{x+z}{x+y+z}\ge \frac{x}{x+y}
\]
i.e. \[
(x+z)(x+y)\ge x(x+y+z)
\]
which is equivalent to:
\[
x(x+y)+z(x+y) \geq x(x+y)+x z
\]
Eliminating common terms, this is equivalent to:
\[
zy \geq 0
\]
which is true by non-negativity of $z$ and $y$. This finishes the proof that:
\[
\mathbb{P}(B\mid A\cup B)\ge \mathbb{P}(B\mid A)
\]
\end{itemize}
\end{proof}
