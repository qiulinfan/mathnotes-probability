\chapter*{Homework 1}



\section*{Problem 1}
Let $n \in \mathbb{N}$.
\begin{itemize}
    \item[(a)] Show that $2^n=\sum_{k=0}^n\binom{n}{k}$. Given that a set of $n$ elements has $2^n$ subsets, what is the combinatorial interpretation of this equality?
    \item[(b)] Show that $$\sum_{k \text { odd }, 0 \leq k \leq n}\binom{n}{k}=\sum_{k \text { even }, 0 \leq k \leq n}\binom{n}{k} $$
    \item[(c)] Show that $$\binom{2 n}{n}=\sum_{k=0}^n\binom{n}{k}^2 $$
\end{itemize}

Hint: You may use $\binom{n}{k}^2=\binom{n}{k}\binom{n}{n-k}$.


\section*{Problem 2}
We roll a fair die three times and record the outcomes $a, b, c \in\{1,2,3,4,5,6\}$. What is the probability that the equation $a x^2+b x+c=0$ does not have solutions in the real numbers?

\section*{Problem 3}
An ant starts at the origin $(0,0)$ on the integer lattice. At each step it moves either one unit to the right or one unit upward, each with probability $\frac{1}{2}$. The ant continues moving until it reaches the point $(205,200)$.\\
What is the probability that the ant visits the point $(105,100)$ at some time during its journey?\\
Hint: Start by counting the number of paths from $(0,0)$ to $(205,200)$.



\section*{Problem 4}
From a lottery containing $n$ tickets numbered $1,2, \ldots, n$, a ticket is drawn, its number is recorded, and then it is returned to the lottery. This process is repeated $k \geq 3$ times. Find the probabilities of the following events:
\begin{itemize}
    \item[(a)] Ticket 1 is selected at least once.
    \item[(b)] Tickets 1, 2, and 3 are each selected at least once.
\end{itemize}


\section*{Problem 5}
In a house, drawer $S_1$ contains 3 gold coins and 3 silver coins, while drawer $S_2$ contains 3 gold coins and 6 silver coins. A thief (in the dark) randomly opens one drawer and then randomly takes two coins from it.
\begin{itemize}
    \item[(a)] What is the probability that both coins are gold?
    \item[(b)] If it is discovered (upon his arrest) that he has stolen two gold coins, what is the probability that he opened drawer $S_1$ ?
\end{itemize}



\section*{Problem 6}
Let $A$ and $B$ be events of a probability space with $\mathbb{P}(A)>0$. Show that:
\begin{itemize}
    \item[(a)] $\mathbb{P}(A \cup B)>0$ and $\mathbb{P}(A \cap B \mid A \cup B) \leq \mathbb{P}(A \cap B \mid A)$.
    \item[(b)] $\mathbb{P}(B \mid B \cup A) \geq \mathbb{P}(B \mid A)$.
\end{itemize}

