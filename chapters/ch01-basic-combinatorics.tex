\chapter{basic combinatorics}
\section{permutations}
\begin{definition}{permutations}
一个 permutation 就是对一组 objects 的一个 \textbf{rearrangement} (这些 objects 中可以有 same 的也可以有 distinct 的).\\
    对于 $n$ 个 \textbf{distinct objects}, 一共存在 \[n! = n(n-1)(n-2)\cdots \] 个 permutations.
\end{definition}

\begin{example}
    求 "STATISTICS" 的 \# distinct permutations.
    \begin{solution}
    这里一共有 10 个 objects. 但问题是: 其中有 3 个 \(S\), 3 个 \(T\), 2 个 \(I\) 是相同的.\\
    于是: 我们首先假设它们都是 distinct 的, 则存在 \(10!\) 个 permutations. 而, 每个 permutation 都包含了对 3 个 $S$ 的一个子 permutation. 而对 3 个 $S$ 的任意 permutation 都是相同的! 同样的道理 apply to 3 个 $T$ 和 2 个 $I$.\\
    所以这个结果是真实结果的 \(3!3!2!\) 倍.\\
    同样地, 由于
    因而, 正确结果是: \[
    \frac{10!}{3!3!2!1!1!}
    \]
    \end{solution}
\end{example}
\begin{remark}
    求一组 objects 的 distinct permutations 的数量时, 对于其中相同的 objects, 我们只需要除去它们的重复数量的 factorial (即: 它们自己内部有多少个 permutation 都算作一个 permutation).\\
    公式: \[
    \frac{n!}{n_1!n_2!\cdots n_k!}
    \]
    其中 $n_1, n_2, \cdots, n_k$ 是每个 object 的重复数量.\\
    这个式子又叫做 multinomial coefficient.
    \begin{definition}{multinomial coefficient}
    令 $n\in \mathbb{N}, n_1, n_2, \cdots, n_k\in \mathbb{N}, n_1 + n_2 + \cdots + n_k = n$, 我们定义 multinomial coefficient:
    \[
    \binom{n}{n_1, n_2, \cdots, n_k} = \frac{n!}{n_1!n_2!\cdots n_k!}
    \]
    \end{definition}

    \begin{proposition}
        如果我们需要 $n_1$ 个 object 1, $n_2$ 个 object 2, $\cdots$, $n_k$ 个 object $k$, 那么它们的 distinct permutations 的数量为:
        \[
        \binom{n_1 + n_2 + \cdots + n_k}{n_1, n_2, \cdots, n_k}
        \]
    \end{proposition}
\end{remark}

\section{combinations}
\begin{definition}{combinations}
一个 combination 就是从一个 set 中选取若干个 elements, 而忽略它们的顺序.
\end{definition}
\begin{proposition}
    从 $n$ 个 \textbf{distinct objects} 中选取 $k$ 个的 combinations 的数量为: \[
\binom{n}{k} = \frac{n!}{k!(n-k)!}
\]
\end{proposition}
\begin{proof}
    我们可以将问题转化为: 从 $n$ 个 \textbf{distinct objects} 中选取 $k$ 个的 permutations 的数量, 然后再除去重复的 permutations.
    而, 从 $n$ 个 \textbf{distinct objects} 中选取 $k$ 个的 permutations 的数量为: \[
    n \times (n-1) \times \cdots \times (n-k+1) = \frac{n!}{(n-k)!}
    \]
    而其中, 对于每个 valid combination, 都包含了它的所有 ordered permutations, 即重复了 $k!$ 次.
    因此, 最终的结果为: \[
    \frac{n!}{k!(n-k)!}
    \]
\end{proof}


\subsection{binomial theorem}

\begin{theorem}{Binomial Theorem}
令 $x,y\in \mathbb{R}, n \in \mathbb{N}$, 则有: \[
(x+y)^n = \sum_{k=0}^n \binom{n}{k} x^{n-k} y^k
\]
\end{theorem}
\begin{proof}
    我们可以 prove this by combiinatorial interpretation. 因为把 $x+y$ 展开即 $n$ 个 $(x+y)$ 的乘积. 即:
    对于每个被乘项, 我们都是在 $x$ 和 $y$ 之间选择一个.\\
    因而: $x^k y^{n-k}$ 的系数就是从 $n$ 个 $(x+y)$ 中选取 $k$ 个 $x$ 的 combinations 的数量, 即 $\binom{n}{k}$.\\
    考虑所有的 possible $k$ 值, 我们得到: \[
    (x+y)^n = \sum_{k=0}^n \binom{n}{k} x^{n-k} y^k
    \]
    这是 combinatorial 的 proof.
\end{proof}
另外一种更轮椅的思路是 prove by induction. 这需要一个辅助的 proposition:
\begin{proposition}
    \[
    \binom{n}{k} = \binom{n-1}{k-1} + \binom{n-1}{k}
    \]
\end{proposition}
这个等式的 combinatorial interpretation 很 trivial: 对于其中的任意一个 object: 
\begin{itemize}
    \item 这个 object 被选中的情况, combinations 的数量: $\binom{n-1}{k-1}$ (从其他里面选 $k-1$ 个);
    \item 这个 object 不被选中的情况, combinations 的数量: $\binom{n-1}{k}$ (从其他里面选 $k$ 个).
\end{itemize}


\subsection*{problems}
\begin{example}
    一个 52-card deck, 取 5 张随机牌, 我们获得:
    \begin{itemize}
        \item 4 张同 rank 的牌, 最后一张不同 rank 的牌
        \item a full house (3 张同 rank 的牌, 2 张同 rank 的牌)
    \end{itemize}
    的概率是多少?
    \begin{solution}
        一共有 $\binom{52}{5}$ 种取法.
        取 4 张同 rank 的牌: 13 种取法.
        取最后一张不同 rank 的牌: 52-4 = 48 种取法.
        因而, 概率是: \[
        \frac{13 \times 48}{\binom{52}{5}} 
        \]
        如果是取 3 张同 rank 的牌 + 两张 different 同 rank 的牌: 我们首先在 4 个花色里面选 3 个, 有 $\binom{4}{3}$ 种取法. \\
        因而选取 3 cards of the same rank 的数量为: $13 \cdot \binom{4}{3}$.\\
        然后选取剩余的两张: 
        然后故技重施, 从剩下的 12 个 rank 里面选 1 个, 而选择它们的花色有 $\binom{4}{2}$ 种取法. 因而, 概率是: \[
        \frac{13 \cdot \binom{4}{3} \cdot 12 \cdot \binom{4}{2}}{\binom{52}{5}} 
        \]
    \end{solution}
\end{example}

\begin{example}
    我们有 $n$ 把钥匙, 其中有一把是正确的. 尝试 $k$ 次, 能够成功开门的概率是多少?
    \begin{solution}
        一共有 $n!$ 种钥匙的 permutations. 我们需要的情况: 正确的钥匙出现在前 $k$ 个位置:
        \begin{itemize}
            \item 正确钥匙出现在第 $1$ 个位置, 其他 $n-1$ 随便排列: $\cdot (n-1)!$ 种
            \item $\cdots$
            \item 正确钥匙出现在第 $k$ 个位置, 其他 $n-1$ 随便排列: $k \cdot (n-1)!$ 种
        \end{itemize}
        因而正确的 permutations 的数量为: \[
        k \cdot (n-1)!
        \]
        因而, 概率是: \[
        \frac{k \cdot (n-1)!}{n!} = \frac{k}{n}
        \]
    \end{solution}
\end{example}
\begin{remark}
    正确钥匙出现在每个位置上的概率都是 $1/n$. 因而它出现在前 $k$ 个位置的概率就是 $k/n$.\\
    这是一类典型的问题: 不放回的抽取. 在只关心"是否在前 $k$ 次成功"这一事件时, 这个过程等价于"正确钥匙在一个随机排列中的位置". 这一结果只和比例有关, 与过程细节无关. 只要没有信息bias, 没有偏好, 完全随机, 那么成功概率只取决于:
    \begin{itemize}
        \item 你允许的尝试次数 $k$
        \item 总可能性数 $n$
    \end{itemize}
    而如果是放回则是不同的情况. 通过补事件容易得:
    \[
    \mathbb{P}(A)=1-\left(1-\frac{1}{n}\right)^k
    \]
    放回和不防回最大的区别是: 放回时, 每次尝试都是独立的, 而不放回时, 每次尝试都不是独立的. 最明显的例子就是, 如果不放回, $k=n$ 时概率为 1. 而如果放回, 不论 $k$ 有多大, 概率都小于 1; 当 $k \ll n$ 的时候, 这两个概率相近 (符合直觉, 因为 $n$ 很大时放回和不放回几乎没区别.) 
\end{remark}

\begin{example}
    一个篮子里有 $10$ 个 red balls 和 \(5\) 个 blue balls. 我们随机从中取出 $3$ 个 balls, exactly 其中 $1$ 个是 blue ball 的概率是多少? 如果每次都放回呢?
    \begin{solution}
    不放回:
        \[
        \mathbb{P}{(\text{exactly one blue ball})} = 
        \binom{10}{2} \binom{5}{1} \big/ \binom{15}{3} = \frac{45}{91}
        \]
        放回: 5 ways to choose the blue ball, 10 ways to choose the red ball, 以及 3 positions to place the blue ball, 
        \[
        \mathbb{P}{(\text{exactly one blue ball})} = 
        3\cdot \frac{3 \cdot 10^2}{15^3}
        \]
    \end{solution}
\end{example}




\section{combinations with repetition}
\begin{definition}{combinations with repetition}
一个 combination with repetition 就是从一个 set 中选取若干个 elements, 而忽略它们的顺序, 并且\textbf{允许重复选取}.
\end{definition}
\begin{proposition}
    从 $n$ 个 \textbf{distinct objects} 中选取 $k$ 个的 combinations with repetition 的数量为: \[
    \binom{n+k-1}{k}
    \]
\end{proposition}
\begin{remark}
    关于这个问题我们最开始可能会犯一个错误: 认为 combinations with $k$ repetitions 的数量为: \[
    \binom{kn}{k}
    \]
    但是想一下就知道这是错的. 因为我们相当于给 repeated 的同一个 objects 赋予了顺序, 从而计入了额外的数量.
\end{remark}
\begin{proof}
    这个问题比较巧妙. 我们上面错误的尝试已经表明: 用 "make copies" 的方法行不通. 
    我们需要变换一下思路. 原问题是 "要选哪几个元素, 每个元素要选几个".
    而我们可以把这个问题理解为: 一共有 $k$ 个位置, $n$ 个组, 我们给每个组分配多少个位置?\\
    Formalize 这个想法即: 对于第 $i$ 个 object, 我们给它分配 $x_i$ 个位置. 所有满足条件的 combinations 可以 represent by:
    \[
    \{ y = (x_1, x_2, \cdots, x_n) \in \mathbb{Z}_{\geq 0}^n : x_1 + x_2 + \cdots + x_n = k \}
    \]
    到这里我们想到一个经典的问题: stars and bars. 即: 把 $k$ 个星星分成 $n$ 个组, 每个组至少有 1 个星星. 这个问题等价于: 把 $k$ 个星星和 $n-1$ 个隔板排成一排, 然后选择 $n-1$ 个隔板的位置.\\
    问题是: 我们这里, 一个组可以有 $0$ 个 stars; 但是这是小问题. 因为我们可以 set $x_i' = x_i + 1$, 问题等价转化为:
    \[
    \{ y = (x_1', x_2', \cdots, x_n') \in \mathbb{Z}_{\geq 1}^n : x_1' + x_2' + \cdots + x_n' = k + n \}
    \]
    这就强制每个组至少有一个 star, 于是可以使用 stars and bars 的方法来解决. 即: 用 $n-1$ 个隔板隔开 $k+n$ 个星星 (有 $k+n-1$ 个空档). 
    因而, 满足条件的 combinations 的数量为: \[
    \binom{k+n-1}{n-1} = \binom{k+n-1}{k}
    \]
\end{proof}
\begin{example}
有 5 种口味的 ice creams. 一个人随机选择 20 个 scoops. 求: 每种口味至少被选中一次的 probability.
\begin{solution}
即从 $5$ 种口味中选取 $20$ 个 combinations with repetition. 于是 sample space 的大小: $\binom{25-1}{20}$.\\
而满足条件的 combinations: 即每种口味我们都预选一个. 然后再从$5$ 种口味中选取 $15$ 个 combinations with repetition. 
\[
\mathbb{P}(\text{each flavor is selected at least once}) = \frac{\binom{19}{15}}{\binom{24}{20}} = \frac{\binom{24}{20}}{\binom{24}{20}}
\]
\end{solution}
\end{example}



\section{inclusion-exclusion principle}
\begin{proposition}{inclusion-exclusion principle}
如果 $\Omega$ 是一个 finite measure space, 那么对于任意 $A_1, \ldots, A_n \subseteq \Omega$ 的, 有:
\[
\left|\cup_{i=1}^n A_i\right|=\sum_{i=1}^n\left|A_i\right|-\sum_{i<j}\left|A_i \cap A_j\right|+\sum_{i<j<k}\left|A_i \cap A_j \cap A_k\right|-\ldots+(-1)^{n+1}\left|A_1 \cap \ldots \cap A_n\right| .
\]
\end{proposition}
\begin{remark}
    这里的 $\sum_{i<j}$ 等 index 意思就是\textbf{考虑所有可能的 combinations}, 不考虑顺序, 和下标的顺序没有关系.
    比如一共有三个集合 $A_1, A_2, A_3$, 那么 $\sum_{i<j}|A_i \cap A_j|$ 就是 $A_1 \cap A_2, A_1 \cap A_3, A_2 \cap A_3$ 的并集.\\
    这个定理是 countable additivity 的直接推广.
\end{remark}

\subsection*{problems}
\begin{example}
(Divisibility) 令 $n \in \mathbb{N}$, 我们随机取一个 $x \in \{1, 2, \cdots, n\}$, 求 $x$ is divisible by 2 or 3 or 5 的概率.
\begin{solution}
令 $A_2, A_3, A_5$ 为 $x$ 是 2, 3, 5 的倍数的 events. 即:
\[
A_i = \{ k \in \{1, 2, \cdots, n\} \mid k \text{ is divisible by } i \}
\]
于是我们要计算的是: 
\begin{align*}
\mathbb{P}(A_2 \cup A_3 \cup A_5) &= \mathbb{P}(A_2) + \mathbb{P}(A_3) + \mathbb{P}(A_5) - \mathbb{P}(A_2 \cap A_3) - \mathbb{P}(A_2 \cap A_5) - \mathbb{P}(A_3 \cap A_5) + \mathbb{P}(A_2 \cap A_3 \cap A_5) \\
&= \frac{\lfloor \frac{n}{2} \rfloor + \lfloor \frac{n}{3} \rfloor + \lfloor \frac{n}{5} \rfloor - \lfloor \frac{n}{6} \rfloor - \lfloor \frac{n}{10} \rfloor - \lfloor \frac{n}{15} \rfloor + \lfloor \frac{n}{30} \rfloor}{n}
\end{align*}
\end{solution}
\end{example}



\begin{example}{(matching problem)}
假设有 \(n\) 个人参加一个 event, 每个人都上交了一顶帽子; 现在再把帽子随机地发给每个人, 求没有人拿回自己的帽子的概率.
\begin{solution}
令 $A_i$ 为第 $i$ 个人拿回自己的帽子的事件. 则我们要求的概率是: $\mathbb{P}(\bigcap_{i=1}^n A_i^c)$.
\begin{align*}
\mathbb{P}(\bigcap_{i=1}^n A_i^c) &= \mathbb{P}((\bigcup_{i=1}^n A_i)^c) \\
&= 1- \mathbb{P}(\bigcup_{i=1}^n A_i)\\
&= 1- \frac{|\bigcup_{i=1}^n A_i|}{n!}
\end{align*}
由于: 
\begin{align*}
|\bigcup_{i=1}^n A_i| &= \sum_{i=1}^n\left|A_i\right|-\sum_{i<j}\left|A_i \cap A_j\right|+\sum_{i<j<k}\left|A_i \cap A_j \cap A_k\right|-\ldots+(-1)^{n+1}\left|A_1 \cap \ldots \cap A_n\right| \\
&= \binom{n}{1}(n-1)!-\binom{n}{2}(n-2)!+\binom{n}{3}(n-3)!-\cdots+(-1)^{n+1}\\
&= n!\sum_{k=1}^n(-1)^{k-1} \frac{1}{k!}
\end{align*}
我们可以得到: 
\[
\mathbb{P}(\bigcap_{i=1}^n A_i^c)  = \sum_{k=2}^n \frac{(-1)^k}{k!}
\]
\end{solution}
\end{example}
\begin{remark}
当 $k\to\infty$ 的时候, 这个结果趋向于 $e^{-1} \approx 0.367879$. (By Taylor's expansion of $e^x$.)
\end{remark}

