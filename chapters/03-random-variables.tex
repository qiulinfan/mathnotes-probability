\chapter{random variable and distribution}

\begin{definition}{random variable}
    对于 probability space \((\Omega, \mathcal{F}, \mathbb{P})\), 一个 random variable 是一个 \textbf{Borel measurable function} $X: \Omega \rightarrow \mathbb{R}$.
\end{definition}

\begin{remark}
    回顾 measurable function 的定义: 即对于任意 Borel set $B \subset \mathbb{R}$, 有 $X^{-1}(B) \in \mathcal{F}$.
    例如: \(X(\omega) = \omega^2\) 是一个 measurable function, 因为对于任意 Borel set $B \subset \mathbb{R}$, 有 $X^{-1}(B) = \{\omega \in \Omega | \omega^2 \in B\} \in \mathcal{F}$.
\end{remark}


% 其 cdf $F_X: \mathbb{R} \rightarrow [0,1]$ 就是 $X$ 这个 real-valued measurable function 对于 $(-\infty,x]$ 的 preimage, 意义是 "随机变量的值小于等于 $x$ 的概率"  
% \[
% F_X (x) = P( X^{-1}(-\infty,x))
% \]
% discrete random variable 的 pmf, 表示每个离散点 $x$ 的 $P(X^{-1}(\{x\}))$.
% 即 \[
% p(x) := P(X^{-1}(\{x\})) = P(X=x)
% \]
% continuous random variable 的 pdf, 表示在某个点上概率分布的密度有多大 \[
% f_X(x) :=  \frac{dF_X(x)}{dx}
% \]
% 或者写作: \[
% f_X(x) :=  \frac{dF_X(x)}{dm}
% \]
% 是这个 $F_X$ 对于 Lebesgue measure $m$ 的 Radon-Nikodym derivative.


% 随机变量的 expectation 是它在这个 prob space 上的积分, 表示它的值的 prob-weighted average
% \[
% E (X):= \int_\Omega X  \;d P
% \]
% 随机变量的 variance 是 $(X-E(X))^2$ 这个 induced 随机变量在这个 prob space 上的积分, 表示原随机变量值离它的值的 weighted average 的聚集程度. \[
% Var(X) = \int_\Omega (X-E(X))^2   \;dP
% \]



\begin{proposition}
    Prob space \((\Omega, \mathcal{F}, \mathbb{P})\) 上的 
    random variables \(X_1, X_2, \ldots, X_k: \Omega \rightarrow \mathbb{R}\),
    任取 Borel measurable function \(g: \mathbb{R}^k \rightarrow \mathbb{R}\), 
    \begin{align*}
        X: \Omega &\rightarrow \mathbb{R} \\
        \omega &\mapsto g(X_1(\omega), X_2(\omega), \ldots, X_k(\omega))
    \end{align*}
    也是一个 random variable.
\end{proposition}
\begin{proof}
    我们在 measure theory 中证明过: 对于任意的 finite seq of Borel measureable functions 
    \( (  {f_i}: \Omega \to \mathbb{R} )_{i=1}^k\), 其各作为一个维度组成的函数
    \(f = (f_1,\cdots, f_k)\) 也是一个 Borel measurable function (from \(\Omega\) 到 \(\mathbb{R}^k\)).\\
    而这里的 \(X\) 就是 \(g \circ f\), 是两个 Borel measurable function 的 composition, 因而也是一个 Borel measurable function (即 random variable).
\end{proof}



\section{distributions}
\begin{definition}{probability distribution}
    对于 random variable \(X: \Omega \to \mathbb{R}\) on \(\Omega, \mathcal{F}, \mathbb{P}\), 
    其 probability distribution 是 \(X\) 对于 \(\mathbb{P}\) 的 pushforward measure, 
    记作 \(P^X: \mathcal{B}(\mathbb{R}) \to [0,1]\). 即
    \[
    P^X(B) := P(X^{-1}(B)), \quad \forall B \in \mathcal{B}(\mathbb{R})
    \]
    我们 write:
    \[
    X \sim P^X
    \]
\end{definition}
\begin{remark}
    看着有点不直观. 实则我们拆解:
    \begin{itemize}
        \item \(X^{-1}(B)\) 即: 有多少 samples (即哪一个事件 \(E\in\mathcal{F}\)) 在这个 RV $X$ 的映射下落入 \(B\) 这个区间?
        \item \(P(X^{-1}(B))\) 即: 这个事件 \(E\) 的概率是多少?
    \end{itemize}
    接下来的定义和例子会更加直观.
\end{remark}
\begin{definition}{distribution function (也称 \textbf{cumulative distribution function, cdf})}
    对于 random variable \(X: \Omega \to \mathbb{R}\), 
    其 \textbf{cumulative distribution function} 是 \(P^X\) 这一函数的 distribution function, 
    记作 \(F_X\). 即对于任意 \(x \in \mathbb{R}\), 有
    \[
    F_X(x) := P^X((-\infty,x]) = P(X^{-1}((-\infty,x]))
    \]
\end{definition}
\begin{remark}
    \(F_X(x)\) 即 \(\mathbb{P}(\{X\leq x\})\) 我们也会简写为 \(\mathbb{P}(X \leq x)\).\\
    \(F_X(x)\) 的意义是: random variable \(X\) 的值不超过 \(x\) 的概率.\\
\end{remark}


\begin{example}{(geometric probability)}
Fix \(a,b >0\). 
我们在正方形区域 \(\Omega = [0,a] \times [0,b]\) 上均匀地随机选取一个点 \((x,y)\).\\
我们 define 随机变量: \(X: \Omega \to \bR\) by \(X(x,y) = x\). 求 \(X\) 的 distribution function \(F_X\).\\
这很简单: 对于 \(x\in [0,a]\),
\[
F_X(x) = \bP(X \leq x) = \frac{m([0,x]\times [0,b])}{m([0,a]\times [0,b])} = 
\frac{x \cdot b}{a \cdot b} = \frac{x}{a}, \quad 0 \le x \le a
\]
因为
\begin{equation}
    F_X(x) = \begin{cases}
    0, & x < 0 \\
    \frac{x}{a}, & 0 \le x \le a \\
    1, & x > a
    \end{cases}
\end{equation}
注意: 在这个例子中, probability measure \(P\) 是 Lebesgue measure on \([0,a]\times [0,b]\). 很多时候我们在
计算 RV 的 distribution function 的时候, 都是直接直观地计算 \(P(X \leq x)\). 但是在稍微复杂一些
的例子中, 我们需要对各个条件的 formalization 更清楚一点.
\end{example}


\begin{proposition}{distribution function 的性质}
    对于 random variable \(X: \Omega \to \mathbb{R}\), 其 distribution function \(F_X\) 满足:
    \begin{itemize}
        \item \(F_X\) 是 non-decreasing (non-strictly increasing) 的.
        \item $\lim _{x \rightarrow+\infty} F_X(x)=1$, $\lim _{x \rightarrow-\infty} F_X(x)=0$.
        \item \(F_X\) 是 right-continuous 的. 即对于任意 \(x_0\), 有 \(\lim_{x \to x_0^+} F_X(x) = F_X(x_0)\).
        \item 对于任意 \(x_0\), 有 \(\lim_{x \to x_0^-} F_X(x) = P(X < x_0)\). 
        并且 \(\bP(X = x_0) = \lim_{x \to x_0^+} F_X(x) - \lim_{x \to x_0^-} F_X(x)\).
        \item 对于任意 \(x_1 < x_2\), 有 \(P(X \in (x_1,x_2]) = F_X(x_2) - F_X(x_1)\).
    \end{itemize}
    
\end{proposition}
\begin{proof}
    其他都显然. right-continuity 是源自 measure 的 continuity from above, 
    我们证明一下: 考虑一个单调递减序列 $\{x_n\}\downarrow x_0$, 
    令 seq of events \(A_n = \{X \leq x_n\}\), 注意这是一个嵌套递减的 set seq. 
    通过 \(\bR\) 的 completeness 容易证明:  \[A_0 = \bigcap_{n=1}^\infty A_n = \{\omega: X(\omega) \leq x_0\}\]
    由 measure 的 continuity from above, \(\bP(A_0) = \lim_{n\to\infty} \bP(A_n)\).
    也即 \(\lim_{x \to x_0^+} F_X(x) = \bP(X \leq x_0)\).\\
    而下面一条 \(\lim_{x \to x_0^-} F_X(x) = \bP(X < x_0)\) 同理是源自 measure 的
     continuity from below.\\
    那么 \(\bP(X = x_0) = \bP(X \leq x_0) - \bP(X < x_0)\) 是 natural 的 (by def).
\end{proof}

\begin{remark}
    我们此时心里会默认一件事情: 一旦知道了 RV \(X\) 的 distribution function 
    \(F_X\), 就知道了 \(X\) 的完整 distribution \(P^X\).
    这个直觉是对的. \\
    回忆一下: 我们在 measure theory 中, 
    证明 Carathéodory extension theorem 的时候, 
    证明过一个 \(\pi-\lambda\) theorem, 以它作为工具才
    把 premeasure 从一个 algebra extend 
    到一个 measure on the generated \(\sigma\)-algebra:
    \begin{theorem}{Dynkin's \(\pi-\lambda\) theorem}
        设 \(\mathcal{P}\) 是一个 \(\pi\)-system (closed under finite intersection), \(\mathcal{L}\) 是一个 \(\lambda\)-system 
        (closed under complementation and countable disjoint union), 
        如果 \(\mathcal{P} \subseteq \mathcal{L}\), 那么 \(\sigma(\mathcal{P}) \subseteq \mathcal{L}\).
    \end{theorem}
    
    我们考虑所有的 half-open interval \((-\infty,x]\) 组成的集合
     \(\mathcal{G}\), 这个集合 generate the Borel \(\sigma\)-algebra, 并且
     还是一个 \(\pi\)-system, 即: 它 closed under finite intersection:
     \[
     (-\infty, a] \cap (-\infty, b] \in \mathcal{G}, \quad \forall a,b \in \mathbb{R}
     \]
     然后考虑 \[
     \mathcal{L} := \{E \in \sigma(\mathcal{G}) : \bP_1(E) = \bP_2(E)\}
     \]
     即两个 measure 一致的 collection of sets.
     容易证明, 这个 \(\mathcal{L}\) 是一个 \(\lambda\)-system, 并且 \(\mathcal{G} \subset \mathcal{L}\).\\
     从而, \[\bB(\bR) =  \sigma(\mathcal{G}) \subseteq \mathcal{L}\] 并且 \(\bP_1(E) = \bP_2(E)\) 
     对于任意 \(E \in \bB(\bR)\).

     因而:
     \begin{corollary}{distribution function \(F_X\) determines distribution \(P^X\)}
         对于 random variable \(X: \Omega \to \mathbb{R}\), 其 distribution function \(F_X\) 确定了 distribution \(P^X\); 即
         如果两个 random variable \(X\) 和 \(Y\) 满足 \(F_X = F_Y\), 那么 \(\bP^X = \bP^Y\).
     \end{corollary}
\end{remark}


\section{discrete random variables}
\begin{definition}{discrete random variable}
    对于 random variable \(X: \Omega \to \mathbb{R}\),
     如果存在一个 countable set \(S \subseteq \mathbb{R}\) 
     使得 \(\bP(X \in S) = 1\), 
     那么 \(X\) 是一个 discrete random variable.
\end{definition}
\begin{remark}
    就是说, \textbf{\(X\) 的 range a.s. 是 countable 的} (允许 null set 上
    的值不在这个 countable set 里, 但是这是 measure theory 
    下的概念, 所以这些都可以忽略不计). 
\end{remark}
\begin{remark}
    对于 discrete random variable \(X\), 
    其 distribution function \(F_X\) 是一个 step function, 
    只有 countable 多个 jump discontinuity, 
    每个 jump 的大小就是 \(X\) 在那个点的 probability mass.\\
    因而, 对于 discrete random variable \(X\), 
    通常考虑其 \textbf{probability mass function (pmf)} \(p_X\) 更加有用:
    \begin{definition}{probability mass function (pmf)}
        对于 discrete random variable \(X\), 其 pmf 定义为
        \[
        p_X(x) := \bP(X = x), \quad \forall x \in \mathbb{R}
        \]
    \end{definition}
    这个 pmf 就是 \(F_X\) 的 jump size function. 值得一提的是, 
    这个 pmf 其实就是 \(X\) 的 distribution \(P^X\) 对于
     counting measure 的 Radon-Nikodym derivative:
     \[
     p_X = \frac{dP_X}{d\mu_{\text{counting}}}
     \]
    我们假设 \(X\) 的 range 是 \(S = \{x_1, x_2, \ldots\}\), 
    那么对于任意 Borel set \(B\),
    \[
    P^X(B) = \sum_{x_i \in B} P(X = x_i) = \sum_{x_i \in B\cap S} p_X(x_i) = \int_A p_X(x) \;d\mu_{\text{counting}}(x)
     \]     
\end{remark}
下面我们给出一些经典的 discrete random variable 的例子, 以及它们的 pmf 和 cdf.

\begin{example}{(\textbf{Bernoulli distribution})}
    
    令 \(\Omega:= \{\omega_1, \omega_2\}\), \(\cF := 2^{\Omega}\).\\
    令 \(\bP(\{\omega_1\}) = p\), \(\bP(\{\omega_2\}) = 1-p\) 来表示这两个 singular 事件的概率.\\
    令 \(X(\omega_1) = 1\), \(X(\omega_2) = 0\) 分别表示 success 和 failure.\\
    那么显然可以计算:
    \[
    \bP(X=0) = \bP(\{\omega_2\}) = 1-p, \quad \bP(X=1) = \bP(\{\omega_1\}) = p
    \]
    我们称 \((\Omega, \mathcal{F}, \bP)\) 为一个 Bernoulli probability space 
    (它 model 了一个 Bernoulli trial, 即一个 random experiment with two possible outcomes: success 和 failure)\\
    称 \(X: \Omega \to \{0,1\}\) 这个 random variable 为一个 \textbf{Bernoulli random variable},
    并称 \(X\) 的 distribution \(P^X\) 为一个 \textbf{Bernoulli distribution}, 写作
    \[
    X \sim \text{Ber}(p)
    \]
    这是最简单的 random variable 和 distribution 了. 它 model 的是: 
    比如我们 toss 一枚 biased coin, 
    以 \(p\) 的概率得到 heads (success), 
    以 \(1-p\) 的概率得到 tails (failure).\\
\end{example}

\begin{example}{(\textbf{Binomial distribution})}

我们 independently repeat \(n\) 次 Bernoulli trial, 
每次 trial 的 success probability 都是 \(p\).\\
考虑:
\[
S := X_1 + X_2 + \cdots + X_n
\] 为 success 的总次数. 那么 \(S\) 是一个 discrete random variable 
\( S: \Omega^n \to \mathbb{Z}_{\geq 0}  \)
容易计算出 \(S\) 的 pmf:
\[
p_S(k) = \bP(S = k) = \binom{n}{k} p^k (1-p)^{n-k}, \quad k = 0,1,\ldots,n
\]
我们称 \(S\) 的 distribution \(P^S\) 为一个 \textbf{Binomial distribution}, 写作
\[
S \sim \text{Bin}(n,p)
\]
它 model 的是: 比如我们 toss 一枚 biased coin \(n\) 次,
得到 heads 的总次数.\\
\end{example}

\begin{example}{(\textbf{Geometric distribution})}

我们 perform independent Bernoulli trial, 每次 trial 的 success probability 都是 \(p\).\\
考虑 random variable \(T\) 表示第一次 success 发生的 trial number. (也就是等于: 我们一直 trial, 直到第一次 success 发生, 那么这个 trial 的 number 就是 \(T\)).\\
那么 \(T\) 是一个 discrete random variable \(T: \Omega^\infty \to \mathbb{Z}_{\geq 1}\).\\
容易计算出 \(T\) 的 pmf:
\[
p_T(k) = \bP(T = k) = (1-p)^{k-1} p, \quad k = 1,2,\ldots
\]
我们称 \(T\) 的 distribution \(P^T\) 为一个 \textbf{Geometric distribution}, 写作
\[
T \sim \text{Geom}(p)
\]
它 model 的是: 比如我们 toss 一枚 biased coin, 
第一次得到 heads 的 trial number.\\
\end{example}


\begin{example}{(\textbf{Negative Binomial distribution})}
这是 Geometric distribution 的 generalization (但不完全一样).\\
我们 perform independent Bernoulli trial, 
每次 trial 的 success probability 都是 \(p\). (即 independent and identically distributed)\\
令 random variable \(T_r\) 表示第 \(r\) 次 success 发生之前
的 failures 的数量.
 (也就是等于: 我们一直 trial, 直到第
\(r\) 次 success 发生, 那么这个 trial 的 number 就是 \(T_r + r\), 
因为 \(T_r\) 是 failures 的数量, 还要加上 \(r-1\) 个 success.
).\\
那么 \(T_r\) 是一个 discrete random variable 
\(T_r: \Omega^\infty \to \mathbb{Z}_{\geq 0}\).\\
容易计算出 \(T_r\) 的 pmf:
\[
p_{T_r}(k) = \bP(T_r = k) = 
\binom{k+r-1}{k}  (1-p)^{k} p^r, \quad k = 0, 1, 2, \ldots
\]
这是因为:最后一次 success 的位置是固定的. 我们要在前 \(k+r-1\) 次 trial 
中选择 \(k\) 次作为 failures.\\
我们称 \(T_r\) 的 distribution \(P^{T_r}\) 为一个 \textbf{Negative Binomial distribution}, 写作
\[
T_r \sim \text{NB}(r,p)
\]  
它 model 的是: 比如我们 toss 一枚 biased coin,
得到第 \(r\) 个 heads 之前会经历的 failures 的数量.\\
\end{example}

\begin{example}{(\textbf{Poisson distribution})}
考虑这一 pmf:
\[
\bP(X=k) = e^{-\lambda} \frac{\lambda^k}{k!}, \quad k = 0, 1, 2, \ldots
\]
我们称 \(X\) 的 distribution \(P^X\) 为一个 \textbf{Poisson distribution}, 写作
\[
X \sim \text{Poi}(\lambda), \quad \lambda > 0
\]
它 model 的是: 在固定时间/空间窗口内, 某类随机事件出现的次数.\\
以 \(\lambda=3\) 为例画出 pmf 示意图:
\begin{center}
\begin{tikzpicture}[x=0.65cm,y=14cm]
    \draw[->] (-0.5,0) -- (10.8,0) node[right] {$k$};
    \draw[->] (0,0) -- (0,0.26) node[above] {$\bP(X=k)$};
    % \lambda=3 时, k=0..10 的 pmf 数值(避免 pgfmath 阶乘/幂运算溢出)
    \foreach \kv/\pv in {
        0/0.04979,
        1/0.14936,
        2/0.22404,
        3/0.22404,
        4/0.16803,
        5/0.10082,
        6/0.05041,
        7/0.02160,
        8/0.00810,
        9/0.00270,
        10/0.00081
    } {
        \fill (\kv,\pv) circle (0.9pt);
    }
    \node[above right] at (5,0.21) {$\lambda=3$};
\end{tikzpicture}
\end{center}
它等 
\end{example}









\section{expectation and variance of random variables}









% discrete random variable 的 pmf, 表示每个离散点 $x$ 的 $P(X^{-1}(\{x\}))$.
% 即 \[
% p(x) := P(X^{-1}(\{x\})) = P(X=x)
% \]
% continuous random variable 的 pdf, 表示在某个点上概率分布的密度有多大 \[
% f_X(x) :=  \frac{dF_X(x)}{dx}
% \]
% 或者写作: \[
% f_X(x) :=  \frac{dF_X(x)}{dm}
% \]
% 是这个 $F_X$ 对于 Lebesgue measure $m$ 的 Radon-Nikodym derivative.