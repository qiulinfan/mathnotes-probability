\chapter{random variables}

\begin{definition}{random variable}
    对于 probability space \((\Omega, \mathcal{F}, \mathbb{P})\), 一个 random variable 是一个 \textbf{Borel measurable function} $X: \Omega \rightarrow \mathbb{R}$.
\end{definition}

\begin{remark}
    回顾 measurable function 的定义: 即对于任意 Borel set $B \subset \mathbb{R}$, 有 $X^{-1}(B) \in \mathcal{F}$.
    例如: \(X(\omega) = \omega^2\) 是一个 measurable function, 因为对于任意 Borel set $B \subset \mathbb{R}$, 有 $X^{-1}(B) = \{\omega \in \Omega | \omega^2 \in B\} \in \mathcal{F}$.
\end{remark}


% 其 cdf $F_X: \mathbb{R} \rightarrow [0,1]$ 就是 $X$ 这个 real-valued measurable function 对于 $(-\infty,x]$ 的 preimage, 意义是 "随机变量的值小于等于 $x$ 的概率"  
% \[
% F_X (x) = P( X^{-1}(-\infty,x))
% \]
% discrete random variable 的 pmf, 表示每个离散点 $x$ 的 $P(X^{-1}(\{x\}))$.
% 即 \[
% p(x) := P(X^{-1}(\{x\})) = P(X=x)
% \]
% continuous random variable 的 pdf, 表示在某个点上概率分布的密度有多大 \[
% f_X(x) :=  \frac{dF_X(x)}{dx}
% \]
% 或者写作: \[
% f_X(x) :=  \frac{dF_X(x)}{dm}
% \]
% 是这个 $F_X$ 对于 Lebesgue measure $m$ 的 Radon-Nikodym derivative.


% 随机变量的 expectation 是它在这个 prob space 上的积分, 表示它的值的 prob-weighted average
% \[
% E (X):= \int_\Omega X  \;d P
% \]
% 随机变量的 variance 是 $(X-E(X))^2$ 这个 induced 随机变量在这个 prob space 上的积分, 表示原随机变量值离它的值的 weighted average 的聚集程度. \[
% Var(X) = \int_\Omega (X-E(X))^2   \;dP
% \]



\begin{proposition}
    Prob space \((\Omega, \mathcal{F}, \mathbb{P})\) 上的 
    random variables \(X_1, X_2, \ldots, X_k: \Omega \rightarrow \mathbb{R}\),
    任取 Borel measurable function \(g: \mathbb{R}^k \rightarrow \mathbb{R}\), 
    \begin{align*}
        X: \Omega &\rightarrow \mathbb{R} \\
        \omega &\mapsto g(X_1(\omega), X_2(\omega), \ldots, X_k(\omega))
    \end{align*}
    也是一个 random variable.
\end{proposition}
\begin{proof}
    我们在 measure theory 中证明过: 对于任意的 finite seq of Borel measureable functions 
    \( (  {f_i}: \Omega \to \mathbb{R} )_{i=1}^k\), 其各作为一个维度组成的函数
    \(f = (f_1,\cdots, f_k)\) 也是一个 Borel measurable function (from \(\Omega\) 到 \(\mathbb{R}^k\)).\\
    而这里的 \(X\) 就是 \(g \circ f\), 是两个 Borel measurable function 的 composition, 因而也是一个 Borel measurable function (即 random variable).
\end{proof}



\section{distributions}
\begin{definition}{probability distribution}
    对于 random variable \(X: \Omega \to \mathbb{R}\), 
    其 probability distribution 是 \(X\) 这个 measurable function 对于 probability measure \(\mathbb{P}\) 的 pushforward measure, 记作 \(P_X\). 即对于任意 Borel set \(B \subset \mathbb{R}\), 有
    \[
    P_X(B) = P(X^{-1}(B))
    \]q
    
\end{definition}