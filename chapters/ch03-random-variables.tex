\chapter{random variables}

\begin{definition}{random variable}
    对于 probability space \((\Omega, \mathcal{F}, \mathbb{P})\), 一个 random variable 是一个 \textbf{measurable function} $X: \Omega \rightarrow \mathbb{R}$.
\end{definition}

\begin{remark}
    这里的 measurable function 是 Borel measurable function, 即对于任意 Borel set $B \subset \mathbb{R}$, 有 $X^{-1}(B) \in \mathcal{F}$.\\例如: \(X(\omega) = \omega^2\) 是一个 measurable function, 因为对于任意 Borel set $B \subset \mathbb{R}$, 有 $X^{-1}(B) = \{\omega \in \Omega | \omega^2 \in B\} \in \mathcal{F}$.
\end{remark}




随机变量 $X: \Omega \rightarrow \mathbb{R}$ 就是一个从样本空间 \((\Omega, \mathcal{F},P)\) 到 $\mathbb{R}$ 的 measurable function.\\

其 cdf $F_X: \mathbb{R} \rightarrow [0,1]$ 就是 $X$ 这个 real-valued measurable function 对于 $(-\infty,x]$ 的 preimage, 意义是 "随机变量的值小于等于 $x$ 的概率"  
\[
F_X (x) = P( X^{-1}(-\infty,x])
\]
discrete random variable 的 pmf, 表示每个离散点 $x$ 的 $P(X^{-1}(\{x\}))$.
即 \[
p(x) := P(X^{-1}(\{x\})) = P(X=x)
\]
continuous random variable 的 pdf, 表示在某个点上概率分布的密度有多大 \[
f_X(x) :=  \frac{dF_X(x)}{dx}
\]
或者写作: \[
f_X(x) :=  \frac{dF_X(x)}{dm}
\]
是这个 $F_X$ 对于 Lebesgue measure $m$ 的 Radon-Nikodym derivative.


随机变量的 expectation 是它在这个 prob space 上的积分, 表示它的值的 prob-weighted average
\[
E (X):= \int_\Omega X  \;d P
\]
随机变量的 variance 是 $(X-E(X))^2$ 这个 induced 随机变量在这个 prob space 上的积分, 表示原随机变量值离它的值的 weighted average 的聚集程度. \[
Var(X) = \int_\Omega (X-E(X))^2   \;dP
\]









