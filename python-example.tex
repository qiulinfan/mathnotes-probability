\documentclass{elegantbook}

\title{代码渲染示例}
\subtitle{ElegantBook中的Python和文本代码渲染}
\author{示例作者}
\date{\today}

\begin{document}

\maketitle

\tableofcontents

\chapter{代码渲染示例}

\section{Python代码环境(黑色背景)}

下面是一个完整的Python代码示例,现在采用黑色背景:

\begin{python}[caption={蒙蒂霍尔问题模拟}, label=code:montyhall]
import random

def monty_hall_sim(trials=10000):
    stay_wins = 0
    switch_wins = 0

    for _ in range(trials):
        # 初始化门
        doors = [0, 0, 0]
        car_position = random.randint(0, 2)
        doors[car_position] = 1
        
        # 玩家选择
        player_choice = random.randint(0, 2)
        
        # 主持人打开门
        possible_host_doors = [
            i for i in range(3) 
            if i != player_choice and doors[i] == 0
        ]
        host_opens = random.choice(possible_host_doors)
        
        # 计算结果
        if doors[player_choice] == 1:
            stay_wins += 1
            
        remaining_door = [
            i for i in range(3) 
            if i != player_choice and i != host_opens
        ][0]
        
        if doors[remaining_door] == 1:
            switch_wins += 1

    print(f"坚持不换胜率: {stay_wins/trials:.2%}")
    print(f"换门胜率: {switch_wins/trials:.2%}")

if __name__ == "__main__":
    monty_hall_sim()
\end{python}

\section{C++代码环境}

下面展示C++代码环境,同样采用黑色背景:

\begin{cpp}[caption={蒙蒂霍尔问题C++实现}, label=code:montyhall-cpp]
#include <iostream>
#include <vector>
#include <random>
#include <algorithm>

class MontyHallSimulator {
private:
    std::random_device rd;
    std::mt19937 gen;
    std::uniform_int_distribution<int> door_dist;
    
public:
    MontyHallSimulator() : gen(rd()), door_dist(0, 2) {}
    
    void simulate(int trials = 10000) {
        int stay_wins = 0;
        int switch_wins = 0;
        
        for (int i = 0; i < trials; ++i) {
            // 初始化门:0代表羊,1代表车
            std::vector<int> doors{0, 0, 0};
            int car_position = door_dist(gen);
            doors[car_position] = 1;
            
            // 玩家最初的选择
            int player_choice = door_dist(gen);
            
            // 主持人打开一扇有山羊的门
            std::vector<int> possible_doors;
            for (int j = 0; j < 3; ++j) {
                if (j != player_choice && doors[j] == 0) {
                    possible_doors.push_back(j);
                }
            }
            
            std::uniform_int_distribution<int> host_dist(0, possible_doors.size() - 1);
            int host_opens = possible_doors[host_dist(gen)];
            
            // 计算结果
            if (doors[player_choice] == 1) {
                stay_wins++;
            }
            
            // 换门后的选择
            int remaining_door = -1;
            for (int j = 0; j < 3; ++j) {
                if (j != player_choice && j != host_opens) {
                    remaining_door = j;
                    break;
                }
            }
            
            if (doors[remaining_door] == 1) {
                switch_wins++;
            }
        }
        
        std::cout << "总实验次数: " << trials << std::endl;
        std::cout << "坚持不换胜率: " << (double)stay_wins / trials * 100 << "%" << std::endl;
        std::cout << "换门胜率: " << (double)switch_wins / trials * 100 << "%" << std::endl;
    }
};

int main() {
    MontyHallSimulator simulator;
    simulator.simulate();
    return 0;
}
\end{cpp}

\section{纯文本环境}

下面展示txt环境,用于显示普通文本、配置文件、数据等:

\begin{txt}[caption={程序运行输出}, label=output:montyhall]
总实验次数: 10000
坚持不换中奖次数: 3331 (概率: 33.31%)
换门后中奖次数: 6669 (概率: 66.69%)
\end{txt}

\section{配置文件示例}

txt环境也可以用来展示配置文件:

\begin{txt}[caption={示例配置文件}, numbers=none]
# 配置文件示例
server.host = localhost
server.port = 8080
database.url = jdbc:mysql://localhost:3306/mydb
database.username = user
database.password = password

# 日志配置
log.level = INFO
log.file = /var/log/app.log
\end{txt}

\section{数据文件示例}

还可以用来展示数据文件内容:

\begin{txt}[caption={CSV数据示例}]
姓名,年龄,城市,分数
张三,25,北京,85.5
李四,30,上海,92.3
王五,28,广州,78.9
赵六,32,深圳,88.7
\end{txt}

\section{命令行/终端环境}

terminal环境用于展示命令行操作,具有黑色背景和白色文字:

\begin{terminal}[caption={运行Python程序}]
$ python montyhall.py
总实验次数: 10000
坚持不换中奖次数: 3331 (概率: 33.31%)
换门后中奖次数: 6669 (概率: 66.69%)

$ python -c "import random; print(random.randint(1,6))"
4
\end{terminal}

也可以使用cmd环境(terminal的别名):

\begin{cmd}[caption={Windows命令提示符}]
C:\Users\qiulin> dir *.py
 驱动器 C 中的卷是 Windows
 卷的序列号是 1234-5678

 C:\Users\qiulin 的目录

2026/01/29  10:30             1,432 montyhall.py
               1 个文件          1,432 字节
               0 个目录              0 可用字节

C:\Users\qiulin> python montyhall.py
总实验次数: 10000
坚持不换中奖次数: 3347 (概率: 33.47%)
换门后中奖次数: 6653 (概率: 66.53%)
\end{cmd}

\section{Unix/Linux命令示例}

\begin{terminal}
$ ls -la
total 24
drwxr-xr-x  3 user user 4096 Jan 29 10:30 .
drwxr-xr-x 15 user user 4096 Jan 29 10:00 ..
-rw-r--r--  1 user user 1432 Jan 29 10:30 montyhall.py
-rw-r--r--  1 user user  256 Jan 29 10:25 requirements.txt

$ cat requirements.txt
numpy>=1.21.0
scipy>=1.7.0
matplotlib>=3.4.0

$ pip install -r requirements.txt
Collecting numpy>=1.21.0
  Using cached numpy-1.24.3-py3-none-any.whl
Installing collected packages: numpy, scipy, matplotlib
Successfully installed matplotlib-3.7.1 numpy-1.24.3 scipy-1.10.1
\end{terminal}

\section{行内代码}

在正文中可以使用行内代码:
- Python代码:\pythoninline{import numpy as np}
- C++代码:\cppinline{std::vector<int> vec{1,2,3};}

\section{简单代码示例}

Python简单示例:
\begin{python}[numbers=none]
# 简单示例  
import numpy as np
data = np.random.normal(0, 1, 1000)
print(f"均值: {np.mean(data):.3f}")
\end{python}

C++简单示例:
\begin{cpp}[numbers=none]
// 简单示例
#include <iostream>
#include <vector>

int main() {
    std::vector<int> numbers{1, 2, 3, 4, 5};
    
    for (const auto& num : numbers) {
        std::cout << num << " ";
    }
    std::cout << std::endl;
    
    return 0;
}
\end{cpp}

\begin{txt}[numbers=none]
这是一个没有行号的文本块
可以用来展示简单的输出
或者其他不需要行号的内容
\end{txt}

\end{document}