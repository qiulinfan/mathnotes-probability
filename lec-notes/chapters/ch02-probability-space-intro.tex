\chapter{Probability space intro}
我们这里跳过所有 measure theory 的内容, 见 notes on measure theory.\\
\begin{definition}{prob space, prob measure, sample space, event space}
    一个 probability space 就是一个 measure space \( (\Omega, \mathcal{F}, \mathbb{P}) \), 其中 $\mathbb{P}(\Omega) = 1$.\\
    对于这样的 measure $\mathbb{P}$, 我们称之为 \textbf{probability measure (概率测度, 即概率)}.\\
    这样的 \(\Omega\), 我们称之为 \textbf{sample space (样本空间)}. \\
    这样的 Simga Algebra \(\mathcal{F}\), 我们称之为 \textbf{event space (事件空间)}. 任意的 $A\in\mathcal{F}$ 我们都称之为一个 \textbf{event (事件)}.
\end{definition}
\begin{remark}
    一个 event 就是这个 prob space 中的一个 measurable set.
    对于 discrete 的 prob space 而言, 我们通常选取 $\mathcal{F} = 2^{\Omega}$, 即 $\Omega$ 的任意子集都是一个 event.
\end{remark}





